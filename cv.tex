\documentclass[utf8,letterpaper,oneside]{article}
% \documentclass[letterpaper,oneside]{ctexart}
% \usepackage[utf8]{inputenc}
\usepackage{setspace}
\usepackage{hyperref}
\usepackage{multicol}
\usepackage{graphicx}
%\usepackage{xeCJK}
%\usepackage{fontspec}
\usepackage{xcolor}
\usepackage{color,soul}
\usepackage{textcomp}
%\setmainfont{kaiu.ttf}
\graphicspath{ {images/}}
\usepackage[left=0.3in, right=0.3in, bottom=0.3in, top=0.3in]{geometry}
\newcommand*{\Skype}{\href{skype:john.smith?add}{john.smith}}
\newcommand{\Absender}[1][\normalsize]{\Skype}
\usepackage[symbol]{footmisc}
\renewcommand{\thefootnote}{\fnsymbol{footnote}}
\pagenumbering{gobble}
\begin{document}
\noindent  \textit{\textbf{Zheye Yuan (Tetsuya Hara)}}%\footnote[1]{\noindent Dual citizenship. \textit{Tetsuya Hara} is on Japanese documents. \textit{Zheye Yuan} is on American documents.}
%\vspace{3ex}
%ine
\small
\begin{center}
 \begin{tabular}{l l}
  Pennsylvania State University                  & \hspace{1in} \href{mailto:ztyh0121@gmail.com}{ztyh0121@gmail.com} or \href{mailto:zxy124@psu.edu}{zxy124@psu.edu} \\
  Department of Statistics                       & \hspace{1in}   \href{http://personal.psu.edu/zxy124/}{personal.psu.edu/zxy124}                                    \\
  325 Thomas Building, University Park, PA 16802 & \hspace{1in}+1 (814) 883-6639                                                                                     \\
 \end{tabular}
\end{center}
%\vspace{-3em}
\noindent
\begin{center}
 \begin{tabular}{l l}
  \hline
                           &                                                                                                                                                  \\
  \textbf{Education}       & \textit{\underline{Pennsylvania State University}, University Park}, {Ph.D., Statistics}, expected May 2019,                                     \\
                           & Advisor: Bing Li, GPA: 3.96/4.00                                                                                                                 \\
                           & \textit{\underline{Hitotsubashi University}, Kunitachi}, B.A., Economics, 2014,                                                                  \\
                           & Advisors: Naoyuki Ishimura, Kenta Kobayashi, GPA: 3.94/4.00                                                                                      \\
                           & \textit{\underline{University of California}, Berkeley}, Exchange, Mathematics, 2013, GPA: 3.814/4.00                                            \\
                           &                                                                                                                                                  \\ \hline
                           &                                                                                                                                                  \\
  \textbf{Work \& Service} & \textit{\underline{{Intern}}, Research \& Development, Ford Global Data Insights \& Analytics}, Summer 2018                                      \\
  \textbf{Experience}      & Clustered car nameplates using hierarchical clustering and Poincar\'e embedding.                                                                 \\
                           & \textit{\underline{{Intern}}, Nomura Research Institute}, Summer 2013                                                                            \\
                           & Analyzed interface \& data storage optimization for portable device used by insurance sales person.                                              \\
                           & \textit{\underline{{Consultant}}, Statistical Consulting Center, Pennsylvania State University}, 2015                                            \\
                           & Provided statistical consulting for graduate students from other departments.                                                                    \\
                           & \underline{\textit{ASA Datafest {Volunteer}}}, Department of Statistics, Pennsylvania State University, 2017                                     \\
                           & \underline{\textit{Dorm Management {Volunteer}}}, International Students Dormitory Association of Kodaira, 2012                                  \\
                           &                                                                                                                                                  
  \\\hline
                           &                                                                                                                                                  \\
  \textbf{Awards, }        & \underline{{\textit{Best Poster Award}}}, Penn State Statistics 50th Anniversary Conference, May 2018                                            \\
  \textbf{Scholarships \&} & \underline{\textit{Graduate Assistanship}}, Department of Statistics, Pennsylvania State University, 2014 to present                             \\
  \textbf{Certification}   & {Graded Ph.D. level} and {taught Junior and Senior level} courses.                                                                               \\
                           & \underline{{\textit{Excellent Academic Achievement}}}, Department of Economics, Hitotsubashi University, 2012                                    \\
  
                           & \underline{{\textit{Japan Student Services Organization Scholarship}}}, for exchange to UC Berkeley, 2013                                        \\
                           & \underline{\textit{Actuarial Exam P}}, 11/17/2011, ID: 82124, \underline{\textit{Actuarial Exam FM}}, 4/9/2012, ID: 66205                        \\
                           &                                                                                                                                                  \\ \hline
                           &                                                                                                                                                  \\
  \textbf{Selected}        & \underline{Asymptotic Distribution of Neural Network Estimator}                                                                                  \\
  \textbf{Research}        & Devised methods to find variance of estimators using second derivative information of a neural network.                                          \\
                           & Reformulated the problem as a likelihood problem to obtain variance from just the first derivative.                                              \\
                           & \underline{Dimension Reduction with Deep Learning}                                                                                               \\
                           & Capitalized on the similarity between minimum average variance estimation (MAVE) and deep learning.                                              \\
                           & Found that the first matrix of weights, as well as other quantities should reside in the dimension reduction                                     \\
                           & space. Simulation study gave results superior to MAVE.                                                                                           \\
                           & \underline{Nonlinear Support Vector Machine (SVM) on Multiple Functional Data}                                                                   \\
                           & Each predictor was a vector of functions, for which we demonstrated how to use the reproducing kernel Hilbert                                    \\
                           & space (RKHS) to catch nonlinear features. Both simulation and application gave superior results.                                                 \\
                           & \underline{Combining Global and Local Dimension Reduction Methods}                                                                               \\
                           & MAVE looks at local structure, while contour regression looks at global structure. Attempted to combine the                                      \\
                           & two for robust performance.                                                                                                                      \\
                           &                                                                                                                                                  \\ \hline
                           &                                                                                                                                                  \\
  \textbf{Selected}        & \underline{Poincar\'e Embedding of Chinese Characters in a Novel}                                                                                \\
  \textbf{Practical}       & Reduced relationship storage space by embedding a graph of proximity of Chinese characters by embedding                                          \\
  \textbf{Experiences}     & them in a Hyperbolic space. This allowed hierarchical structure to be captured more effectively. The algorithm                                   \\
                           & was modified suitably to be implemented in car nameplate classification as well.                                                                 \\
                           & \underline{Other Miscellany}                                                                                                                     \\
                           & Semiprametric Copula Estimation, Quantile Regression, Non-stationary Time Series Analysis,                                                       \\
                           & Discrete Cosine Transform Portfolio Construction, Linear \& Quadratic Programming,                                                               \\
                           & Numerical Calculation of Derivative Prices, Substitute Charge Method, Finite Element Method,                                                     \\
                           & {Reinforcement Learning}, {Gradient Boosting}                                                                                                    \\
                           &                                                                                                                                                  \\ \hline
                           &                                                                                                                                                  \\
  % \\
  \textbf{Notable}         & Statistical Computing, {High Dimensional} Modelling \& Applications, {Nonparametric} Methods,                                                    \\
  \textbf{Coursework}      & Stochastic Processes \& Monte Carlo Methods, Categorical Data, Regression Models, Multivariate Analysis                                          \\
                           &                                                                                                                                                  \\ \hline
                           &                                                                                                                                                  \\
  \textbf{Languages}       & \{{\texttt{python}}\} (used in internship and research), \{{\texttt{julia}, \texttt{r}}\} (used in research),                                    \\ &\{{\texttt{c}, \texttt{c++}}, \texttt{matlab}\} (previously used in research), \{\LaTeX, \texttt{html}, \texttt{css}, {\texttt{git}}, \texttt{UNIX}\} (used daily),                  \\
                           & \{{\texttt{sql}}\} (have some experience), \{Japanese $>$ English $>$ Chinese $>$ French $>$ Spanish\}.                                          \\
                           & Packages I am familiar with in \texttt{python} include \texttt{numpy}, \texttt{pandas}, \texttt{tensorflow}, \texttt{pytorch}, \texttt{chainer}, \\
                           & \texttt{keras}, \texttt{gensim}, etc.                                                                                                            
 \end{tabular}
\end{center}
\end{document}
