\documentclass[utf8,letterpaper,oneside]{article}
% \documentclass[letterpaper,oneside]{ctexart}
% \usepackage[utf8]{inputenc}
\usepackage{setspace}
\usepackage{hyperref}
\usepackage{multicol}
\usepackage{graphicx}
%\usepackage{xeCJK}
\usepackage{fontspec}
\usepackage{xcolor}
\usepackage{color,soul}
\usepackage{textcomp}
%\setmainfont{kaiu.ttf}
\graphicspath{ {images/}}
\usepackage[left=0.3in, right=0.3in, bottom=0.29in, top=0.29in]{geometry}
\newcommand*{\Skype}{\href{skype:john.smith?add}{john.smith}}
\newcommand{\Absender}[1][\normalsize]{\Skype}
\usepackage[symbol]{footmisc}
\renewcommand{\thefootnote}{\fnsymbol{footnote}}
\pagenumbering{gobble}
\begin{document}
\noindent  {\setmainfont{kaiu.ttf}\large 原晣也}
%\vspace{3ex}
%\hline
\small
\setmainfont[BoldFont=ipaexg.ttf]{ipaexm.ttf}

\begin{center}
 \begin{tabular}{l l}
  郵便番号16802                                        & \hspace{1in} \href{mailto:teghenn@gmail.com}{teghenn@gmail.com}                \\
  アメリカ合衆国ペンシルバニア州  ユニバーシティパーク & \hspace{1in}   \href{http://personal.psu.edu/zxy124/}{personal.psu.edu/zxy124} \\
  ペンシルバニア州立大学トーマス棟325室                & \hspace{1in}+1 (814) 883-6639                                                  \\
 \end{tabular}
\end{center}
%\vspace{-3em}
\noindent
\begin{center}
 \begin{tabular}{l l}
  \hline
                      &                                                                                                \\
  \textbf{学歴}       & ペンシルバニア州立大学---統計学---博士---2019年5月卒業予定                                     \\
                      & 指導教官:李兵(Bing Li)---GPA: 3.96/4.00                                                     \\
                      &                                                                                                \\
                      & 一橋大学---経済学---学士---2014年7月卒業                                                       \\
                      & 指導教官:石村直之、小林健太---GPA: 3.94/4.00                                                  \\
                      &                                                                                                \\
                      & カリフォルニア大学バークレー校---数学---交換留学---2013年8月から2014年5月まで                  \\
                      & GPA: 3.814/4.00                                                                                \\
                      &                                                                                                \\ \hline
                      &                                                                                                \\
  \textbf{職歴}       & フォードGDIA研究開発部門インターン---2018年夏                                                  \\
  \textbf{その他経験} & 階層クラスタリング手法とポアンカレ埋め込みを用いて車種を分類                                   \\
                      &                                                                                                \\
                      & 野村総合研究所インターン---2013年夏                                                            \\
                      & 保険外商員が使用する携帯機器に入力できるデータ種類と保険契約の際必要となる情報が合致するか確認 \\
                      &                                                                                                \\
                      & ペンシルバニア州立大学統計諮問センターコンサルタント---2015年                                  \\
                      & 他学部の研究者に統計的アドバイスを提供                                                         \\
                      &                                                                                                \\
                      & アメリカ統計学会データフェスボランティア---2017年、2018年                                      \\
                      & 参加チームに助言提供、大会運営補佐                                                             \\
                      &                                                                                                \\
                      & 一橋大学小平国際学生宿舎ボランティア---2012年                                                  \\
                      & 寮運営支援                                                                                     \\
                      &                                                                                                
  \\\hline
                      &                                                                                                \\
  \textbf{表彰}       & ペンシルバニア州立大学50周年記念学会最優秀ポスター賞---2018年5月                               \\
  \textbf{奨学金}
                      &                                                                                                \\
  \textbf{資格}       & ペンシルバニア州立大学統計学部大学院生奨学金---2014年4月から2019年5月まで                      \\
                      & 博士課程レベルの授業の宿題の採点、学部上級生レベルの授業の講師                                 \\
                      &                                                                                                \\
                      & 一橋大学経済学部学業成績優秀者表彰---2012年                                                    \\
                      &                                                                                                \\
                      & カリフォルニア大学バークレー校留学JASSO奨学金(給付型)---2013年                               \\
                      &                                                                                                \\
                      & アメリカアクチュアリ試験P---2011年11月17日---ID: 82124                                         \\
                      & アメリカアクチュアリ試験FM---2012年4月9日---ID: 66205                                          \\
                      &                                                                                                \\ \hline
                      &                                                                                                \\
  \textbf{研究}       & 充分次元削減、関数データ解析、サポートベクターマシン、ニューラルネットワーク、                 \\
                      & グラフ埋め込み、ポアンカレ埋め込み、再生核ヒルベルト空間                                       \\
                      &                                                                                                \\ \hline
                      &                                                                                                \\
  \textbf{実務経験}   & セミパラメトリックコピュラ推定、区分回帰、非線形時系列解析、強化学習、                         \\
                      & 離散コサイン変換ポートフォリオ組成、線形及び二次計画、勾配ブースティング                       \\
                      & 金融デリバティブの数値計算(代用電荷法、有限要素法)                                           \\
                      &                                                                                                \\ \hline
                      &                                                                                                \\
  % \\
  \textbf{教育背景}   & 統計コンピューティング、高次元モデリングと応用、ノンパラメトリック手法、                       \\
                      & 確率過程とモンテカルロ法、カテゴリカルデータ、回帰手法、統計理論、                             \\
                      & 漸近理論、確率論、多変量解析                                                                   \\
                      &                                                                                                \\ \hline
                      &                                                                                                \\
  \textbf{言語}       & \texttt{python}(インターンシップと研究で使用)                                           \\
                      & \texttt{julia}, \texttt{r}(研究で使用)                                                  \\
                      & \texttt{c}, \texttt{c++}, \texttt{matlab}(学部時代に使用)                               \\
                      & \LaTeX, \texttt{html}, \texttt{css}, \texttt{git}, \texttt{UNIX}(日常的に使用)          \\
                      & \texttt{sql}(経験あり)                                                                  \\
                      & 日本語 $>$ 英語 $>$ 中国語 $>$ フランス語 $>$ スペイン語                                       \\
 \end{tabular}
\end{center}
\end{document}
