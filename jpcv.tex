\documentclass[utf8,letterpaper,oneside]{article}
% \documentclass[letterpaper,oneside]{ctexart}
% \usepackage[utf8]{inputenc}
\usepackage{setspace}
\usepackage{hyperref}
\usepackage{multicol}
\usepackage{graphicx}
\usepackage{xeCJK}
\usepackage{fontspec}
\usepackage{xcolor}
\usepackage{color,soul}
\usepackage{textcomp}
\setmainfont{kaiu.ttf}
\graphicspath{ {images/}}
\usepackage[left=0.3in, right=0.3in, bottom=0.29in, top=0.29in]{geometry}
\newcommand*{\Skype}{\href{skype:john.smith?add}{john.smith}}
\newcommand{\Absender}[1][\normalsize]{\Skype}
\usepackage[symbol]{footmisc}
\renewcommand{\thefootnote}{\fnsymbol{footnote}}
\pagenumbering{gobble}
\begin{document}
\setmainfont{kaiu.ttf}
\noindent  {\setmainfont{kaiu.ttf}\large 原晣也}
%\vspace{3ex}
%\hline
\small
\begin{center}
 \begin{tabular}{l l}
  アメリカ合衆国ペンシルバニア州  ユニバーシティパーク町 & \hspace{1in} \href{mailto:teghenn@gmail.com}{\texttt{teghenn@gmail.com}}                \\
  ペンシルバニア州立大学トーマス棟\texttt{325}室         & \hspace{1in}   \href{http://personal.psu.edu/zxy124/}{\texttt{personal.psu.edu/zxy124}} \\
  郵便番号\texttt{16802}                                 & \hspace{1in}\texttt{+1 (814) 883-6639}                                                  \\
 \end{tabular}
\end{center}
%\vspace{-3em}
\noindent
\begin{center}
 \begin{tabular}{l l}
  \hline
             &                                                                                                                \\
  学歴       & ペンシルバニア州立大学、統計学、博士、\texttt{2019}年\texttt{5}月卒業予定                                      \\
             & 指導教官:李兵\texttt{(Bing Li)}、\texttt{GPA: 3.96/4.00}                                                      \\
             &                                                                                                                \\
             & 一橋大学、経済学、学士、\texttt{2014}年\texttt{7}月卒業                                                        \\
             & 指導教官:石村直之、小林健太、\texttt{GPA: 3.94/4.00}                                                          \\
             &                                                                                                                \\
             & カリフォルニア大学バークレー校、数学、交換留学、\texttt{2013}年\texttt{8}月から\texttt{2014}年\texttt{5}月まで \\
             & \texttt{GPA: 3.814/4.00}                                                                                       \\
             &                                                                                                                \\ \hline
             &                                                                                                                \\
  職歴       & フォードGDIA研究開発部門インターン、2018年夏                                                                   \\
  その他経験 & 階層クラスタリング手法とポアンカレ埋め込みを用いて車種を分類しました                                           \\
             &                                                                                                                \\
             & 野村総合研究所インターン、2013年夏                                                                         \\
             & 保険外商員が使用する携帯機器に入力できるデータ種類と保険契約の際必要となる情報が合致するか確認しました         \\
             &                                                                                                                \\
             & ペンシルバニア州立大学統計諮問センターコンサルタント、2015年                                               \\
             & 他学部の研究者に統計的アドバイスを提供                                                                         \\
             &                                                                                                                \\
             & アメリカ統計学会データフェスボランティア2017年2018年                                                           \\
             &                                                                                                                \\
             & 一橋大学小平国際学生宿舎運営ボランティア2012年                                                             \\
             &                                                                                                                
  \\\hline
             &                                                                                                                \\
  表彰       & ペンシルバニア州立大学50周年記念学会最優秀ポスター賞2018年5月                                                \\
  奨学金
             &                                                                                                                
  ペンシルバニア州立大学統計学部大学院生奨学金2014年4月から2019年5月まで                                            \\
  資格       & 博士課程レベルの授業の宿題の採点をしたり、学部上級生レベルのの授業の講師などを担当しました                     \\
  
             & 一橋大学経済学部学業成績優秀者表彰2012年                                                                   \\
  
             & カリフォルニア大学バークレー校留学JASSO奨学金 2013年                                                           \\
             & アメリカアクチュアリ試験P、2011年11月17日、ID: 82124                                                           \\
             & アメリカアクチュアリ試験FM、2012年4月9日、ID: 66205                                                            \\
             &                                                                                                                \\ \hline
             &                                                                                                                \\
  研究       & 充分次元削減, 関数データ解析, サポートベクターマシン,ニューラルネットワーク,                                   \\
             & グラフ埋め込み, ポアンカレ埋め込み, 再生核ヒルベルト空間                                                       \\
             &                                                                                                                \\ \hline
             &                                                                                                                \\
  実務経験   & セミパラメトリックコピュラ推定、区分回帰、非線形時系列解析,                                                    \\
             & 離散コサイン変換ポートフォリオ組成、線形及び二次計画、                                                         \\
             & 金融デリバティブの数値計算(代用電化法、有限要素法)                                                           \\
             & 強化学習、勾配ブブースティング                                                                                 \\
             &                                                                                                                \\ \hline
             &                                                                                                                \\
  % \\
  教育背景   & 統計コンピューティング, 高次元モデリングと応用, ノンパラメトリック手法                                         \\
             & 確率過程とモンテカルロ法, カテゴリカルデータ, 回帰手法, 統計理論,                                              \\
             & 前金理論, 確率論, 線形手法, 実験計画                                                                          \\
             & 統計推論, 多変量解析                                                                                           \\
             &                                                                                                                \\ \hline
             &                                                                                                                \\
  言語       & \hl{\texttt{python}} (インターンシップと研究で使用)                                                            \\
             & \hl{\texttt{julia}, \texttt{r}} (研究で使用)                                                                   \\
             & \hl{\texttt{c}, \texttt{c++}}, \texttt{matlab} (学部時代に研究で使用)                                          \\
             & \LaTeX, \texttt{html}, \texttt{css}, \hl{\texttt{git}}, experience with \texttt{UNIX} (日常的に使用)           \\
             & \hl{\texttt{sql}} (経験あり)                                                                                   \\
             & 日本語 $>$ 英語 $>$ 中国語 $>$ フランス語 $>$ スペイン語                                                       \\
 \end{tabular}
\end{center}
\end{document}
